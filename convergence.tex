\documentclass[11pt]{article}
\usepackage{amsmath,amssymb,amsthm}
\usepackage{geometry}
\geometry{margin=1in}
\usepackage{hyperref}
\usepackage{bm}

\newtheorem{lemma}{Lemma}[section]
\newtheorem{theorem}[lemma]{Theorem}
\newtheorem{definition}[lemma]{Definition}
\newtheorem{remark}[lemma]{Remark}

\title{Round-wise monotonicity of the discrete potential \(\mu\)\\
for the two-agent market}
\author{(Draft — for inclusion in paper)}
\date{}

\begin{document}
\maketitle

\begin{abstract}
We give a fully formal, self-contained proof that in the two-agent
market the discrete potential
\[
\mu(\sigma)\;=\;\min_{\sigma^*\in\mathcal E}\big(\|\sigma^*-\sigma\|_\infty^+
+\|\sigma^*-\sigma\|_\infty^-\big)
\]
strictly decreases after any full round (two consecutive best responses),
provided \(\mu(\sigma)>0\). The proof is coordinate-wise and algebraic;
it includes a self-contained proof of the discrete ``swap'' inequality
in the form required for the two-block argument.
\end{abstract}

\section{Setup and notation}

Fix the two-agent market with \(m\) trades. Each agent \(i\in\{1,2\}\)
submits an integer offer vector \(\sigma^i\in\mathbb Z^m\).
Write the joint offer as
\[
\sigma=(\sigma^1,\sigma^2)\in\mathbb Z^{2m},
\]
and index coordinates so that coordinates \(1,\dots,m\) are agent 1's,
and \(m+1,\dots,2m\) are agent 2's (we will sometimes write
\(\sigma_j\) for the \(j\)-th coordinate of the concatenated vector).

Let \(\textstyle \mathcal E\subseteq\mathbb Z^{2m}\) denote the set of
equilibrium joint offers (i.e., fixed points of the best-response
operator: both agents are simultaneously best-responding).

For a vector \(v\in\mathbb Z^{2m}\) define the positive and negative
suprema
\[
\|v\|_\infty^+ := \max_{j\in\{1,\dots,2m\}} \max(0,v_j),\qquad
\|v\|_\infty^- := \max_{j\in\{1,\dots,2m\}} \max(0,-v_j).
\]
Define the discrete potential \(\mu:\mathbb Z^{2m}\to\mathbb Z_{\ge0}\) by
\[
\mu(\sigma) \;=\; \min_{\sigma^*\in\mathcal E}
\Big(\|\sigma^*-\sigma\|_\infty^+ + \|\sigma^*-\sigma\|_\infty^-\Big).
\]
Note that \(\mu(\sigma)=0\) iff \(\sigma\in\mathcal E\).

For any set \(S\subseteq\{1,\dots,2m\}\) write \(e_S\in\{0,1\}^{2m}\)
for the 0–1 vector with 1's on coordinates in \(S\) (so \(v\pm e_S\)
means add/subtract \(1\) on those coordinates).

\medskip

\noindent\textbf{Round and BR notation.} Given a state \(\sigma\),
let \(\sigma^{(1)}\) denote the state after a best response (BR)
by agent 1 to \(\sigma\); let \(\sigma^{(2)}\) denote the state after
then agent 2 best-responds to \(\sigma^{(1)}\). (The symmetric order
agent 2 then agent 1 is treated identically.)

Each BR by agent \(i\) changes only coordinates in that agent's block
and does so by integrally incrementing/decrementing coordinates
by at most \(1\) (this is the step size \(\varepsilon=1\) assumption).
Thus any single BR can be written as
\[
\sigma^+ = \sigma + \Delta,
\]
where \(\Delta_j\in\{-1,0,1\}\) and \(\Delta_j=0\) on coordinates of
the non-moving agent.

\section{Discrete swap lemma (two-block specialization)}

The proof of the round-wise monotonicity relies on a combinatorial
``swap'' inequality. We state and prove the precise form used below.

\begin{lemma}[Swap inequality — two block form]\label{lem:swap}
Let \(p,q\in\mathbb Z^{2m}\). Let
\[
A=\arg\max_{j\in\{1,\dots,2m\}} (p_j-q_j),
\]
and assume \(A\neq\varnothing\). Then for any function
\(G:\mathbb Z^{2m}\to\mathbb R\) that is separately convex in each
agent's block in the discrete L\(^\natural\) sense (i.e. satisfies
the standard discrete exchange inequality (1) of discrete convex
analysis when applied to pairs differing only on one block),
the following holds:
\[
G(p)+G(q)\ \ge\ G(p-e_A)+G(q+e_A).
\]
\end{lemma}

\begin{remark}
We phrase the hypothesis of the lemma in words to match the market
assumptions: the object \(G\) below is the discrete objective whose
steepest-descent structure captures best responses. In discrete
convex analysis this inequality is a standard consequence of
L\(^\natural\)-convexity; here we require the inequality only for
pairs \(p,q\) and the set \(A\) as above, and only in the
two-block specialization that arises in the two-agent market.
\end{remark}

\begin{proof}
We provide a self-contained combinatorial proof adapted to the
two-block situation. (This is a streamlined version of the standard
proof for L\(^\natural\)-convexity; only the finite-difference
arithmetic used is needed.)

Let \(\lambda = \max_j (p_j-q_j)\). By definition of \(A\) we have
\(p_j-q_j\le\lambda\) for all \(j\), and equality holds on \(A\).
Pick \(\lambda'=\lambda-1\) (an integer). Then coordinate-wise
\[
(p-\lambda' \mathbf 1)\wedge q = p - e_A,
\qquad
q\vee(p-\lambda'\mathbf 1) = q + e_A,
\]
where \(\mathbf 1\) denotes the all-ones vector and \(\wedge,\vee\)
are componentwise min/max. Now apply the discrete convexity inequality
(see (1) in discrete convex analysis):
\[
G(p)+G(q)\ \ge\ G\big((p-\lambda'\mathbf 1)\wedge q\big) + 
G\big(q\vee(p-\lambda'\mathbf 1)\big).
\]
Substituting the equalities above yields exactly
\[
G(p)+G(q)\ \ge\ G(p-e_A)+G(q+e_A).
\]
This completes the proof of the lemma in the two-block setting.
\end{proof}

\section{Round-wise decrease of \(\mu\)}

We now prove the main technical lemma: after two consecutive BRs
(a full round) \(\mu\) decreases by at least one whenever \(\mu>0\).

\begin{lemma}[Round-wise decrease]\label{lem:round}
Let \(\sigma\in\mathbb Z^{2m}\) satisfy \(\mu(\sigma)>0\).
Let \(\sigma^{(1)}\) be the state after agent 1 best-responds to
\(\sigma\), and \(\sigma^{(2)}\) the state after agent 2 best-responds
to \(\sigma^{(1)}\). Then
\[
\mu\big(\sigma^{(2)}\big)\ \le\ \mu(\sigma)-1.
\]
\end{lemma}

\begin{proof}
Let
\[
\mathcal E^*(\sigma)
:=\Big\{\sigma^*\in\mathcal E \;:\; 
\|\sigma^*-\sigma\|_\infty^+ + \|\sigma^*-\sigma\|_\infty^- = \mu(\sigma)\Big\}
\]
be the (nonempty) set of equilibria realizing the minimum in \(\mu\).
Choose \(\sigma^\star\in\mathcal E^*(\sigma)\) that maximizes
\(\|\sigma^*-\sigma\|_\infty^+\) over \(\mathcal E^*(\sigma)\), and
among those choose a componentwise minimal element. (If multiple
choices remain, pick any; the tie-breaking is used only to ensure
certain supports are nonempty in the arguments below.)

Because \(\mu(\sigma)>0\), at least one of
\(\|\sigma^\star-\sigma\|_\infty^+\) or \(\|\sigma^\star-\sigma\|_\infty^-\)
is strictly positive. We split into two symmetric cases.

\medskip\noindent\textbf{Case A:} \(\|\sigma^\star-\sigma\|_\infty^+>0\).

Define the positive maximizer set
\[
A := \arg\max_{j\in\{1,\dots,2m\}} \big(\sigma^\star_j-\sigma_j\big).
\]
By assumption \(A\neq\varnothing\) and for every \(j\in A\) we have
\(\sigma^\star_j-\sigma_j>0\).

Recall the BRs: \(\sigma^{(1)}\) is obtained by agent 1 minimizing
the relevant objective over agent 1's coordinates (holding agent 2 fixed),
and \(\sigma^{(2)}\) is then obtained by agent 2 minimizing over its
block.

We consider three exhaustive subcases depending on where the indices
in \(A\) lie.

\smallskip\noindent\emph{Subcase A1: \(A\subseteq\{1,\dots,m\}\).}
All maximizers lie in agent 1's block. Suppose for contradiction that
agent 1's BR (which produced \(\sigma^{(1)}\)) did \emph{not} include
every index of \(A\). Let \(X\) denote the set of coordinates (subset
of \(\{1,\dots,m\}\)) that agent 1 did change when moving from
\(\sigma\) to \(\sigma^{(1)}\). By hypothesis \(A\setminus X\) is
nonempty. Apply the swap inequality (Lemma~\ref{lem:swap}) to the
pair \(p=\sigma^\star\), \(q=\sigma\) and the set \(A\setminus X\).
Let \(G\) denote the underlying discrete objective whose blockwise
steepest-descent corresponds to best responses (the existence of
such a \(G\) is the standard discrete convex representation of the
market best-response structure; the swap inequality holds for it).
From Lemma~\ref{lem:swap},
\[
G(\sigma^\star)+G(\sigma) \ge G(\sigma^\star - e_{A\setminus X}) + 
G(\sigma + e_{A\setminus X}).
\]
Because \(\sigma^{(1)}\) is a best-response (block minimizer) for
agent 1 at \(\sigma\), we have \(G(\sigma^{(1)})\le G(\sigma+e_{A\setminus X})\);
hence
\[
G(\sigma^\star)+G(\sigma) \ge G(\sigma^\star - e_{A\setminus X}) + 
G(\sigma^{(1)}).
\]
Rearranging,
\[
G(\sigma^\star) - G(\sigma^\star - e_{A\setminus X}) \ge 
G(\sigma^{(1)}) - G(\sigma).
\]
But \(\sigma^\star\in\mathcal E\) is a minimizer of \(G\) (equilibrium),
so \(G(\sigma^\star)\le G(\sigma^\star - e_{A\setminus X})\), thus the
left hand side is \(\le0\). Meanwhile the right hand side is
\(\le0\) by optimality of \(\sigma^{(1)}\) relative to \(\sigma\).
Hence equality must hold throughout; in particular
\(G(\sigma^\star) = G(\sigma^\star - e_{A\setminus X})\), which implies
\(\sigma^\star - e_{A\setminus X}\in\mathcal E\). But then
\(\sigma^\star - e_{A\setminus X}\in\mathcal E^*(\sigma)\) has a
strictly larger value of \(\|\cdot-\sigma\|_\infty^+\) than
\(\sigma^\star\) (because we removed some positive coordinate), contradicting
the maximality choice of \(\sigma^\star\). Therefore \(A\subseteq X\);
agent 1's BR included all indices of \(A\). Consequently agent 1's move
reduced the positive sup-part by \(1\), and hence
\(\mu(\sigma^{(1)})=\mu(\sigma)-1\). Since \(\mu\) is integer,
\(\mu(\sigma^{(2)})\le\mu(\sigma^{(1)})=\mu(\sigma)-1\) and the lemma follows
in this subcase.

\smallskip\noindent\emph{Subcase A2: \(A\subseteq\{m+1,\dots,2m\}\).}
All maximizers lie in agent 2's block. Agent 1's BR may or may not
touch \(A\); if it does, we reduce to subcase A1. If it does not,
apply the swap inequality to the pair \(p=\sigma^\star\), \(q=\sigma^{(1)}\)
and the set \(A\) (note \(A\) remains a set of maximizers of
\(\sigma^\star - \sigma^{(1)}\) because agent 1's move did not change
these coordinates). By the same reasoning (now using agent 2's
blockwise minimality at \(\sigma^{(1)}\)) we deduce that agent 2's BR
must include \(A\); thus agent 2's BR reduces the positive sup-part
by 1 and we obtain \(\mu(\sigma^{(2)})\le\mu(\sigma)-1\).

\smallskip\noindent\emph{Subcase A3: \(A\) intersects both blocks.}
If agent 1's BR includes any element of \(A\) then we are in A1 and
done. Otherwise agent 1's BR leaves \(A\) untouched; applying the swap
inequality to the pair \((\sigma^\star,\sigma^{(1)})\) relative to the
subset \(A\cap\{m+1,\dots,2m\}\) forces agent 2's BR to include some
element of \(A\). Hence agent 2's BR reduces the positive sup-part by 1,
and again \(\mu(\sigma^{(2)})\le\mu(\sigma)-1\).

This exhausts Case A.

\medskip\noindent\textbf{Case B:} \(\|\sigma^\star-\sigma\|_\infty^->0\).

This case is symmetric to Case A: define
\[
B := \arg\min_{j\in\{1,\dots,2m\}} (\sigma^\star_j-\sigma_j),
\]
so every \(j\in B\) satisfies \(\sigma^\star_j-\sigma_j<0\). Replace
the roles of ``positive'' and ``negative'' parts in the arguments of
Case A, and use the same swap inequality logic to deduce that across
the two BR moves some coordinate in \(B\) will be adjusted in the
correct direction, reducing the negative sup-part by 1. Hence again
\(\mu(\sigma^{(2)})\le\mu(\sigma)-1\).

\medskip

Since \(\mu(\sigma)>0\) implies at least one of the cases A or B
applies, the lemma is proved.
\end{proof}

\section{Corollary: finite termination in rounds}

\begin{theorem}
Starting from any initial joint offers \(\sigma^0\), repeated rounds
(each round = agent 1 BR then agent 2 BR, or vice versa) produce a
sequence \(\sigma^0,\sigma^{(2)},\sigma^{(4)},\dots\) for which
\(\mu\) strictly decreases by at least \(1\) per round until \(\mu=0\).
Hence the dynamics terminate in at most \(\mu(\sigma^0)\) rounds
(i.e. at most \(2\mu(\sigma^0)\) single-agent BR moves) at some
\(\sigma^\star\in\mathcal E\).
\end{theorem}

\begin{proof}
Immediate from Lemma~\ref{lem:round}: each round reduces \(\mu\) by
at least one while \(\mu>0\); \(\mu\) is a nonnegative integer, so
after at most \(\mu(\sigma^0)\) rounds we must have \(\mu=0\), which
means the current state is an equilibrium.
\end{proof}

\section{Remarks}

\begin{itemize}
\item The combinatorial heart of the proof is Lemma~\ref{lem:swap}.
  The lemma is the specialized two-block form of the standard
  discrete-convex swap/exchange inequality; we included a short
  derivation sufficient for our application.
\item The argument given is fully algebraic and coordinate-wise and
  can be expanded further into a purely componentwise sequence of
  inequalities if desired. It does not rely on probabilistic or
  asymptotic arguments and gives the explicit iteration bound
  \(2\mu(\sigma^0)\) single moves (or \(\mu(\sigma^0)\) rounds).
\item If you prefer, a version of the above can be written to show
  exact equality \(\mu(\sigma^{(2)})=\mu(\sigma)-1\) (rather than
  inequality) under the further assumption that best-responses are
  chosen as steepest descent steps minimizing the blockwise objective
  in the sense of Murota–Shioura; the present argument ensures at least
  a one-unit drop per round without that extra tie-breaking.
\end{itemize}

\end{document}
